\documentclass[11pt, a4paper]{article}
\usepackage[utf8]{inputenc}
\usepackage[margin=.7in]{geometry}
\usepackage{listings}
\usepackage{setspace}
\usepackage{xcolor}
\usepackage{tcolorbox}
\usepackage{titlesec}
\usepackage{enumitem}
\usepackage{amssymb}
\usepackage{amsmath}
\usepackage{bm}
\usepackage{multicol}
\usepackage{fancybox}
\usepackage{graphicx}
\graphicspath{{./Figures/}}
\usepackage{color}
\usepackage{hyperref}
\hypersetup{
	colorlinks=true,
	linkcolor=blue,
	urlcolor=purple,
}
\titleformat*{\section}{\LARGE\bfseries\filcenter}
\titleformat*{\subsection}{\Large\bfseries}
\titleformat*{\subsubsection}{\large\bfseries}
\definecolor{codegreen}{rgb}{0,0.5,0.3}
\definecolor{codegray}{rgb}{0.5,0.5,0.5}
\definecolor{codered}{rgb}{0.78,0,0}
\definecolor{codepurple}{rgb}{0.58,0,0.68}
\definecolor{backcolour}{rgb}{0.95,0.95,0.92}
\lstdefinestyle{Pystyle}{
	language = Python,
    backgroundcolor=\color{backcolour},   
    commentstyle=\color{gray},
    keywordstyle=\color{black},
    numberstyle=\tiny\color{codepurple},
    stringstyle=\color{codered},
    basicstyle=\ttfamily\footnotesize,
    breakatwhitespace=false,         
    breaklines=true,                 
    captionpos=b,                    
    keepspaces=true,                 
    numbers=left,                    
    numbersep=5pt,                  
    showspaces=false,                
    showstringspaces=false,
    showtabs=false,                  
    tabsize=2,
    morekeywords = {as},
    keywordstyle = \color{codegreen}
}
\lstset{style=Pystyle}
\tcbset{
	colbacktitle=red!50!white, 
	title=Example, 
	coltitle=black, 
	colback=white, 
	fonttitle=\bfseries
}

\begin{document}
	\begin{titlepage}
		\begin{center} \Huge \textbf{TensorFlow Developer Certificate Notes} \end{center}
		\tableofcontents
		\newpage
	\end{titlepage}
	
%%%% PAGE 1 %%%%
	
	\section{Fundamentals}
	\begin{itemize}
		\item \textbf{tf.constant()} is not mutable, but \textbf{tf.Variable()} is by using the \textit{.assign()} method on the var object.
		\item You must set both the global \textbf{tf.random.set\_seed()} and function \textbf{seed=} parameter to get reproducible results for shuffle function.
		\item We can \textit{add dimensions} to a tensor whilst keeping the same information (\textit{newaxis} and \textit{expand\_dims} have same output).
	\begin{lstlisting}
	rank_3_tensor = rank_2_tensor[..., tf.newaxis] # "..." means "all dims prior to"
	rank_2_tensor, rank_3_tensor # shape (2, 2), shape (2, 2, 1) 
	tf.expand_dims(rank_2_tensor, axis=-1) # "-1" means last axis (2, 2, 1)  \end{lstlisting}
		\item \textbf{tf.reshape()} will change the shape in the order they appear (top left to bottom right) and \textbf{tf.transpose()} simply flips the matrix.
		\item We can reduce tensor sizes in memory by changing the datatype (i.e. float32 cast to float16). 
		\item We can perform aggregation on tensors by using \textbf{reduce()\_[action]} and using min, max, mean, sum, etc. We can also find positional arguments using \textbf{tf.argmin()} or \textbf{tf.argmax()}.
		\item 
	\end{itemize}
	
	
	
\end{document}