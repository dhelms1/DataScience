\documentclass[11pt, a4paper]{article}
\usepackage[utf8]{inputenc}
\usepackage[margin=.7in]{geometry}
\usepackage{listings}
\usepackage{setspace}
\usepackage{xcolor}
\usepackage{titlesec}
\usepackage{enumitem}
\usepackage{amssymb}
\usepackage{amsmath}
\usepackage{bm}
\usepackage{multicol}
\usepackage{graphicx}
\graphicspath{{./Figures/}}
\usepackage{color}
\usepackage{hyperref}
\hypersetup{
	colorlinks=true,
	linkcolor=blue,
	urlcolor=blue,
}
\titleformat*{\section}{\LARGE\bfseries\filcenter}
\titleformat*{\subsection}{\Large\bfseries}
\titleformat*{\subsubsection}{\large\bfseries}
\definecolor{codegreen}{rgb}{0,0.5,0}
\definecolor{codegray}{rgb}{0.5,0.5,0.5}
\definecolor{codered}{rgb}{0.78,0,0}
\definecolor{codepurple}{rgb}{0.58,0,0.68}
\definecolor{backcolour}{rgb}{0.95,0.95,0.92}
\lstdefinestyle{Pythonstyle}{
	language = Python,
    backgroundcolor=\color{backcolour},   
    commentstyle=\color{gray},
    keywordstyle=\color{codegreen},
    numberstyle=\tiny\color{codegray},
    stringstyle=\color{codered},
    basicstyle=\ttfamily\footnotesize,
    breakatwhitespace=false,         
    breaklines=true,                 
    captionpos=b,                    
    keepspaces=true,                 
    numbers=left,                    
    numbersep=5pt,                  
    showspaces=false,                
    showstringspaces=false,
    showtabs=false,                  
    tabsize=2,
    morekeywords = {as},
    keywordstyle = \color{codegreen}
}
\lstset{style=Pythonstyle}

\begin{document}
	\begin{titlepage}
		\begin{center} \Huge \textbf{DeepLearning.AI TensorFlow Developer} \end{center}
		\tableofcontents
		\newpage
	\end{titlepage}
%%%% PAGE 1 %%%%

	\begin{spacing}{1.1}
	\section{Introduction to TensorFlow for AI, ML, and DL}
	\subsubsection{Callbacks}
	We can use \textbf{callbacks} in order to stop training when we reach a certain accuracy we desire. This is to stop the loss from beginning to increase again if we start to overfit the model. \href{https://www.tensorflow.org/api_docs/python/tf/keras/callbacks/Callback}{Click here} to see the TensorFlow Callbacks documentation.
	\begin{lstlisting}
	import tensorflow as tf
	print(tf.__version__)
	
	class myCallback(tf.keras.callbacks.Callback):
		def on_epoch_end(self, epoch, logs={}):
			if(logs.get('accuracy')>0.6): # might need to use 'acc' instead
				print("\nReached 60% accuracy so cancelling training!")
				self.model.stop_training = True
	
	callbacks = myCallback()
	
	mnist = tf.keras.datasets.fashion_mnist
	(x_train, y_train),(x_test, y_test) = mnist.load_data()
	x_train, x_test = x_train / 255.0, x_test / 255.0
	
	model = tf.keras.models.Sequential([
		tf.keras.layers.Flatten(),
		tf.keras.layers.Dense(512, activation=tf.nn.relu),
		tf.keras.layers.Dense(10, activation=tf.nn.softmax)
	])
	
	model.compile(optimizer='adam', 
	              loss='sparse_categorical_crossentropy',
	              metrics=['accuracy'])
	model.fit(x_images, y_labels, epochs=10, callbacks=[callbacks]) \end{lstlisting}\vspace*{1mm}

	\subsubsection{Upload Custom Images}
	We can use the below code to upload a custom image and use it on a trained model.
	\begin{lstlisting}
	import numpy as np
	from google.colab import files
	from keras.preprocessing import image
	
	uploaded = files.upload()
	
	for fn in uploaded.keys():
		# predicting images
		path = '/content/' + fn
		img = image.load_img(path, target_size=(300, 300))
		x = image.img_to_array(img)
		x = np.expand_dims(x, axis=0)
		
		images = np.vstack([x])
		classes = model.predict(images, batch_size=10)
		print(classes[0])
		if classes[0]>0.5:
			print(fn + " is a human")
		else:
			print(fn + " is a horse") \end{lstlisting} \newpage
%%%% PAGE 2 %%%%

	\subsubsection{ImageDataGenerator}
	\begin{lstlisting}
	import tensorflow as tf
	import os
	import zipfile
	from os import path, getcwd, chdir
	from tensorflow.keras.optimizers import RMSprop
    from tensorflow.keras.preprocessing.image import ImageDataGenerator	
	
	# Import and extract zip file containing images
	path = f"{getcwd()}/../tmp2/happy-or-sad.zip"
	zip_ref = zipfile.ZipFile(path, 'r')
	zip_ref.extractall("/tmp/h-or-s")
	zip_ref.close()
	
	def train_happy_sad_model():		
		DESIRED_ACCURACY = 0.999
		
		class myCallback(tf.keras.callbacks.Callback):
			def on_epoch_end(self, epoch, logs={}):
				if(logs.get('acc')>DESIRED_ACCURACY):
					print('\nReached 100% accuracy so stopping training.')
					self.model.stop_training = True
		
		callbacks = myCallback()
		
		# Define and Compile the Model.
		model = tf.keras.models.Sequential([
				tf.keras.layers.Conv2D(64, (3,3), activation='relu', input_shape=(150,150,3)),
				tf.keras.layers.MaxPooling2D(2,2),
				tf.keras.layers.Conv2D(32, (3,3), activation='relu'),
				tf.keras.layers.MaxPooling2D(2,2),
				tf.keras.layers.Conv2D(16, (3,3), activation='relu'),
				tf.keras.layers.MaxPooling2D(2,2),
				tf.keras.layers.Flatten(),
				tf.keras.layers.Dense(512),
				tf.keras.layers.Dense(1, activation='sigmoid')
		])
		
		model.compile(optimizer=RMSprop(lr=0.001),
		              loss='binary_crossentropy',
		              metrics=['accuracy'])
		
		# Create an instance of an ImageDataGenerator 		
		train_datagen = ImageDataGenerator(rescale=1./255)
		
		train_generator = train_datagen.flow_from_directory(
				'/tmp/h-or-s',
				target_size=(150,150),
				class_mode='binary'
		)

		history = model.fit(
				train_generator,
				epochs=50,
				callbacks=[callbacks],
				verbose=1
		)
		
		return history.history['acc'][-1] \end{lstlisting}\newpage
%%%% PAGE 3 %%%%

	\section{CNN's in TensorFlow}
	
	
	
	
	
	
	
	
	
	
	
	
	
	
	
	
	
	
	
	
	
	
	
	
	
	
	
	
	
	
\end{spacing}
\end{document}