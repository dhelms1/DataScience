\documentclass[11pt, a4paper]{article}
\usepackage[utf8]{inputenc}
\usepackage[margin=.7in]{geometry}
\usepackage{listings}
\usepackage{setspace}
\usepackage{xcolor}
\usepackage{titlesec}
\usepackage{enumitem}
\usepackage{amssymb}
\usepackage{amsmath}
\usepackage{bm}
\usepackage{multicol}
\usepackage{graphicx}
\graphicspath{{./Figures/}}
\usepackage{color}
\usepackage{hyperref}
\hypersetup{
	colorlinks=true,
	linkcolor=blue,
	urlcolor=blue,
}
\titleformat*{\section}{\LARGE\bfseries\filcenter}
\titleformat*{\subsection}{\Large\bfseries}
\titleformat*{\subsubsection}{\large\bfseries}
\definecolor{codegreen}{rgb}{0,0.5,0}
\definecolor{codegray}{rgb}{0.5,0.5,0.5}
\definecolor{codered}{rgb}{0.78,0,0}
\definecolor{codepurple}{rgb}{0.58,0,0.68}
\definecolor{backcolour}{rgb}{0.95,0.95,0.92}
\lstdefinestyle{mystyle}{
    backgroundcolor=\color{backcolour},   
    commentstyle=\color{gray},
    keywordstyle=\color{codegreen},
    numberstyle=\tiny\color{codegray},
    stringstyle=\color{codered},
    basicstyle=\ttfamily\footnotesize,
    breakatwhitespace=false,         
    breaklines=true,                 
    captionpos=b,                    
    keepspaces=true,                 
    numbers=left,                    
    numbersep=5pt,                  
    showspaces=false,                
    showstringspaces=false,
    showtabs=false,                  
    tabsize=2,
    morekeywords = {as},
    keywordstyle = \color{codegreen}
}
\lstset{style=mystyle}

\begin{document}
\begin{spacing}{1.1}
	\section{Programming in Python}
	\subsection{Git Bash \& Workflow}
	We will use BASH as our command line interface. Note that BASH is the default shell on Max OS X (so we just use the terminal). On Windows, we will use Git Bash as our shell. We can do the following basics commands: \vspace*{1mm} \\
	\hspace*{1.5mm} $\cdot$ \textbf{ls} - lists the files and folders (also known as directories) inside the current directory. \\
	\hspace*{1.5mm} $\cdot$ \textbf{pwd} - this prints the working directory that you are currently in. \\
	\hspace*{1.5mm} $\cdot$ \textbf{cd} - allows us to change directories, takes argument of desired directory (.. moves previous directory). \\
	\hspace*{1.5mm} $\cdot$ \textbf{mkdir} - this makes a new directory in the current one, takes argument of new directory name. \\
	\hspace*{1.5mm} $\cdot$ \textbf{touch} - this creates a new file in the working directory, takes argument of new file name. \\
	\hspace*{1.5mm} $\cdot$ \textbf{echo} - this lets us add text to a specified file, for example: echo ``Testing" $>>$ test.txt \\
	\hspace*{1.5mm} $\cdot$ \textbf{cat} - this lets us print the contents of a specified file to the terminal, for example: cat test.txt \vspace*{2mm}\\
	A \textit{filesystem} organizes the computer's files and directories into a tree structure. \\
	Note: Using the `up arrow' on the keyboard will allow you to cycle through previous commands. \\~\\
	We can use Git to keep track of changes made to a project over time. A Git project can be thought of having the following workflow: \vspace*{1mm} \\
	\hspace*{1.5mm} 1) \textit{Working Directory} - where you do all the work (creating, editing, deleting, organizing). \\
	\hspace*{1.5mm} 2) \textit{Staging Area} - where you list changes made to working directory (ready to commit). \\
	\hspace*{1.5mm} 3) \textit{Repository} - where Git stores changes as different version of the project. \\~\\
	We can use the following commands in our Git project: \\
	\hspace*{1.5mm} $\cdot$ \textbf{git init} - this will initialize an empty Git repository in your current work directory. \\
	\hspace*{1.5mm} $\cdot$ \textbf{git status} - this will show status of changes (changes to be committed and untracked files). \\
	\hspace*{1.5mm} $\cdot$ \textbf{git add} - this will add a file to staging area, pass a parameter of the filename. \\
	\hspace*{1.5mm} $\cdot$ \textbf{git diff} - shows us the lines added since our last `git add', pass filename parameter (marked by \textcolor{codegreen}{+}). \\
	\hspace*{1.5mm} $\cdot$ \textbf{git commit -m `` "} - permanently stores changes from staging area (pass message in `` "). \\
	\hspace*{1.5mm} $\cdot$ \textbf{git log} - this lets you refer back to earlier versions of a project (store chronologically). \\
	\subsection{Lists}
	We can use \textbf{zip()} to create pairs from multiple lists. However, it returns the location in memory and must be converted back to a \textbf{list()} in order to print it. We can add a single element to a list using \textbf{.append()}, which will place at the end of the list. We can add multiple lists together by using $\bm{+}$ to concatenate them.
	\begin{lstlisting}[language=Python]
	last_semester_gradebook = [("politics", 80), ("latin", 96), ("dance", 97), 
	("architecture", 65)]
	
	subjects = ["physics", "calculus", "poetry", "history"]
	grades = [98, 97, 85, 88]
	subjects.append("computer science")
	grades.append(100)
	gradebook = list(zip(subjects,grades)) # combine and cast as a list
	gradebook.append(("visual arts", 93)) # append a tuple
	print(gradebook)
	
	full_gradebook = gradebook + last_semester_gradebook
	print(full_gradebook) \end{lstlisting}\vspace*{1mm}
%%%% PAGE 2 %%%%
	We can create an array of integers for a given size by \textbf{range()}, which generates starting at a point (0 by default) to the (input value - 1). However, you must convert it to a list since it returns on object.
	\begin{lstlisting}[language=Python]
	my_list = range(9) # values 0 to 8
	my_list_2 = range(5, 15, 3) # start at 5, end at 14, increment by 3
	print(list(my_list_2)) # [5, 8, 11, 14] \end{lstlisting}\vspace*{1mm}
	We can select a section of a list by using syntax array[start:stop], called \textbf{slicing}. 
	\begin{lstlisting}[language=Python]
	suitcase = ['shirt', 'shirt', 'pants', 'pants', 'pajamas', 'books']
	start = suitcase[:3] # same as suitcase[0:3]
	end = suitcase[-2:] # gets last 2 elements of suitcase \end{lstlisting}\vspace*{1mm}
	We can count how many times an element appears in a list with \textbf{.count()}
	\begin{lstlisting}[language=Python]
	votes = ['Jake', 'Jake', 'Laurie', 'Laurie', 'Laurie', 'Jake']
	jake_votes = votes.count('Jake')
	print(jake_votes) \end{lstlisting}\vspace*{1mm}
	We can sort a list alphabetically or numerically with \textbf{.sort()} - only alters a list, doesn't return a value \\
	We can use \textbf{sorted()} to also sort a list, but it will not affect the original list (returns sorted copy)
	\begin{lstlisting}[language=Python]
	games = ['Portal', 'Minecraft', 'Pacman', 'Tetris', 'The Sims', 'Pokemon']
	
	games_sorted = sorted(games)
	print(games) # in same order as above
	print(games_sorted) # new list of sorted games
	
	games.sort()
	print(games) # now the games list is also sorted \end{lstlisting}\vspace*{1mm}
	\textbf{Tuples} are immutable (can't change any values after creating) and are denoted with ( ) \\
	We use tuples to store data that belongs together and don't need order or size to change
	\begin{lstlisting}[language=Python]
	my_info = ('Derek', 22, 'Student')
	name, age, occupation = my_info # will assign each value to a varaible
	
	one_element_tuple = (4,) # NOTE: we need the , after 4 otherwise it wont be a tuple
	one_element_tuple_2 = (4) # same as one_element_tuple_2 = 4	 \end{lstlisting} \vspace*{4mm}	
	\subsection{Loops}
	We can use \textbf{for} loops to iterate through each item in a list, with the following general formula \\
	\hspace*{3mm} - We can use range() to execute a for loop from start (0 by default) to stop (n-1) \\
	\hspace*{3mm} - We can use \textit{break} to exit a for loop when a certain value is found \\
	\hspace*{3mm} - We can use \textit{continue} to move to the next index in a list if a condition is found \\
	If we have a list made of multiple lists, we use \textbf{nested} loops to iterate through them
	\begin{lstlisting}[language=Python]
	sales_data = [[12, 17, 22], [2, 10, 3], [5, 12, 13]]
	scoops_sold = 0
	
	for location in sales_data: # for each list in list
	for sales in location: # for each element in inner list
	scoops_sold += sales
	
	print(scoops_sold) \end{lstlisting} \newpage
%%%% PAGE 3 %%%%
	\noindent We can use \textbf{list comprehension} to efficiently iterate through a list instead of a for loop \\
	We can also use this to alter values in a list and create a new list
	\begin{lstlisting}[language=Python]
	heights = [161, 164, 156, 144, 158, 170, 163, 163, 157] # in cm's
	
	can_ride_coaster = [cm for cm in heights if cm > 161]
	print(can_ride_coaster) # [164, 170, 163, 163] 
	
	celsius = [0, 10, 15, 32, -5, 27, 3] # degrees in C
	
	fahrenheit = [f_temp * (9/5) + 32 for f_temp in celsius] # convert C to F degrees
	print(fahrenheit) # [32.0, 50.0, 59.0, 89.6, 23.0, 80.6, 37.4] \end{lstlisting}\vspace*{4mm}	
	\subsection{List Comprehension / Lambda Functions}
	We can iterate through lists within lists with the following syntax
	\begin{lstlisting}[language=Python]
	nested_lists = [[4, 8], [15, 16], [23, 42]]
	
	product = [(val1 * val2) for (val1, val2) in nested_lists]
	print(product) # [32, 240, 966]	
	
	greater_than = [ (val1 > val2) for (val1, val2) in nested_lists]
	print(greater_than) # [False, False, False] \end{lstlisting}\vspace*{1mm}
	We can iterate through two lists in one list comprehension by using the zip() function.
	\begin{lstlisting}[language=Python]
	x_values_1 = [2*index for index in range(5)] # [0.0, 2.0, 4.0, 6.0, 8.0] 
	x_values_2 = [2*index + 0.8 for index in range(5)] # [0.8, 2.8, 4.8, 6.8, 8.8] 
	
	x_values_midpoints = [(x1 + x2)/2.0 for (x1, x2) in zip(x_values_1, x_values_2)]
	# [0.4, 2.4, 4.4, 6.4, 8.4]	
	
	names = ["Jon", "Arya", "Ned"]
	ages = [14, 9, 35]
	
	users = ["Name: " + n + ", Age: " + str(a) for (n,a) in zip(names,ages)]
	print(users) # ['Name: Jon, Age: 14', 'Name: Arya, Age: 9', 'Name: Ned, Age: 35'] \end{lstlisting}\vspace*{4mm}
	\subsection{Python Objects}
	\subsubsection{Strings}
	
	
	
	
	
	
	
	
\end{spacing}
\end{document}